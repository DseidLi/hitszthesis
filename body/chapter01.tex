% !TEX root = ../main.tex

% 中英标题:\chapter{中文标题}[英文标题]
\chapter{绪\hspace{1em}论}[Introduction]
% 绪论也需像摘要、目录格式那样空出两个半角字符。

\section{课题背景及研究的目的和意义}[Background, objective and significance of the subject]

% 正文内容,注意LaTeX分段有两种方法,直接空一行或者使用<\par>
% 默认首行缩进,不需要在代码编辑区手动敲空格
发展国防工业、微电子工业等尖端技术需要精密和超精密的仪器设备,精密仪器设备要求高速、\dots\dots

\dots\dots

\section{气体润滑轴承及其相关理论的发展概况}[Developmental of gas-lubricated bearing and correlated theories]

气体轴承是利用气膜支撑负荷或减少摩擦的机械构件。\dots\dots

\dots\dots

\subsection{气体润滑轴承的发展}[Developmental of gas-lubricated bearing]

1828年,R.R.Willis$^{[3]}$发表了一篇关于小孔节流平板中压力分布的文章,这是有记载的研究气体润滑的最早文献。\dots\dots

根据间隙内气膜压力的产生原理,气体轴承可以分为四种基本形式:

(1)气体静压轴承:加压气体经过节流器进入间隙,在间隙内产生压力气膜使物体浮起的气体轴承,\dots\dots

\subsubsection{气体润滑轴承的分类}[Classification of gas-lubricated bearing]

\dots\dots

\subsubsection{多孔质气体静压轴承的研究}[Research on porous externally pressurized gas bearing]

由于气体的压力低和可压缩性,\dots\dots。

\section{本文的主要研究内容}[Main research contents of this subject]

本课题的研究内容主要是针对局部多孔质止推轴承的多孔质材料的渗透
率、静压轴承的静态特性、稳定性及其影响因素进行展开,\dots\dots。
