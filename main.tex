%%=============================================
% !Mode:: "TeX:UTF-8"
% !TEX program  = XeLaTeX
%%=============================================
% 模板名称:hitszthesis
% 模板版本:V3.2.8
% 模板作者:杨敬轩(Jingxuan Yang)
% 联系作者:yangjx20@mails.tsinghua.edu.cn & yanglatex2e@gmail.com
% 模板交流:QQ群:1039392552,加群请备注LaTeX、hitszthesis相关说明
% 模板适用:哈尔滨工业大学(深圳)本、硕、博学位论文
% 模板编译:手动编译方法参看 README.md 或 hitszthesis.pdf
%          GNU make 工具:make thesis
%          Windows 批处理脚本:双击 compile.bat 自动编译论文
%          更多编译细节详见说明文档:hitszthesis.pdf
% 更新时间:2023/05/09
% 模板帮助:请**务必务必务必**阅读 hitszthesis.pdf 说明文档,文档查看方法:
%          cmd 命令行:texdoc hitszthesis
%          推荐前往模板的 GitHub 仓库获取最新文件,地址:
%          https://github.com/YangLaTeX/hitszthesis
%%=============================================

% 设置文档类别为 <hitszthesis>
% \documentclass[type=doctor]{hitszthesis}
% \documentclass[type=master]{hitszthesis}
\documentclass[type=bachelor,infoleft=true]{hitszthesis}

% 模板提供以下选项,各个选项之间不要有空格
% 1. type=bachelor|master|doctor
%   含义:本科、硕士、博士学位论文,不设默认值,**必填**
% 2. covertitletworow=true|false
%   含义:本科封面第一页标题单行或多行显示,默认为单行显示(false)
% 3. infoleft=true|false
%   含义:本科封面第二页下划线内容居中或居左显示,默认为居中显示(false)
% 4. mathfont=newtxmath|mtprotwolite|mtprotwo
%   含义:正文数学字体选项:newtxmath(默认),mtprotwolite(lite版,免费),
%         mtprotwo(完全版,需购买授权),
%         mtpro2字体官网:https://www.pctex.com/mtpro2.html
% 5. boldcaption=true|false
%   含义:图表题注是否加粗,默认为不加粗(false)
% 6. tocfour=true|false
%   含义:是否添加第四级目录,只对本科文科个别要求四级目录有效,默认不添加(false)
% 7. fulltime=true|false
%   含义:是否全日制,非全日制如同等学力等,要在coverinformation中设置类型,
%        默认是全日制(true)
% 8. subtitle=true|false
%   含义:论文题目是否含有副标题,默认没有副标题(false)
% 9. openright=true|false
%   含义:博士论文是否要求章节首页必须在奇数页,默认否(false)
% 10. library=true|false
%   含义:是否为提交到图书馆的电子版,默认否(false)
% 11. alphappendix=true|false
%   含义:本科毕业设计附录章节编号是否为大写字母,默认是(true)
% 12. bsfrontpagenumberline=true|false
%   含义:本科毕业设计前言部分页码是否添加短横线,默认是(true)
% 13. bsmainpagenumberline=true|false
%   含义:本科毕业设计正文部分页码是否添加短横线,默认是(true)

% 自定义设置与额外加载的宏包请写在 \file{hitszthesis.sty} 里
\usepackage{hitszthesis}

% 图片存放路径,在这些文件夹里的图片可以直接使用图片文件名调用
\graphicspath{{figures/}{pictures/}}

%%=============================================
% 开始写论文
% !!注意本文仅作为排版格式示例,并不作为毕业论文规范
\begin{document}

% 开始写前言部分
\frontmatter

% 封面信息填写
% !TEX root = ../main.tex

% 专硕请取消下面注释
% \makeatletter
%   \def\hitsz@csubjecttitle{类别}
% \makeatother

\hitszsetup{
  %******************************
  % 注意:
  %   1. 配置里面不要出现空行
  %   2. 不需要的配置信息可以删除
  %******************************
  %
  %=====
  % 秘级
  %=====
  statesecrets={公开},
  natclassifiedindex={TM301.2},
  intclassifiedindex={62-5},
  %
  %=========
  % 中文信息
  %=========
  ctitleone={局部多孔质气体静压轴承},%本科生封面使用
  ctitletwo={关键技术的研究},%本科生封面使用
  ctitlecover={局部多孔质气体静压轴承关键技术的研究},%放在封面中使用,自由断行
  ctitle={局部多孔质气体静压轴承关键技术的研究},%放在原创性声明中使用
  csubtitle={一条副标题}, %一般情况没有,可以注释掉
  cxueke={工学},
  cpostgraduatetype={学术},
  csubject={机械设计制造及其自动化},
  % csubject={机械工程},
  caffil={机电工程与自动化学院},
  % caffil={哈尔滨工业大学(深圳)},
  cauthor={杨敬轩},
  csupervisor={某某某 教授},
  % cassosupervisor={某某某 教授}, % 副导师
  % ccosupervisor={某某某 教授}, % 合作导师
  % 日期自动使用当前时间,若需指定按如下方式修改:
  cdate={2022年6月},
  % 指定第二页封面的日期,即答辩日期
  cdatesecond={2022年06月09日},
  cstudentid={SZ160310217},
  cstudenttype={同等学力人员}, %非全日制教育申请学位者
  %(同等学力人员)、(工程硕士)、(工商管理硕士)、
  %(高级管理人员工商管理硕士)、(公共管理硕士)、(中职教师)、(高校教师)等
  %
  %
  %=========
  % 英文信息
  %=========
  etitle={Research on key technologies of partial porous externally pressurized gas bearing},
  esubtitle={This is the sub title},
  exueke={Engineering},
  esubject={Mechanical Engineering},
  eaffil={Harbin Institute of Technology, Shenzhen},
  eauthor={Jingxuan Yang},
  esupervisor={Prof. XXX},
  % eassosupervisor={XXX},
  % ecosupervisor={Prof. XXX}, % Co-Supervisor off Campus
  % 日期自动生成,若需指定按如下方式修改:
  edate={June, 2022},
  estudenttype={Master of Engineering},
  %
  % 关键词用“英文逗号”分割
  ckeywords={\TeX, \LaTeX, CJK, 论文模板, 毕业论文},
  ekeywords={\TeX, \LaTeX, CJK, hitszthesis, thesis},
}

% 中文摘要
\begin{cabstract}

  摘要是论文内容的高度概括,应具有独立性和自含性,即不阅读论文的全文,就能获得必要的信息。摘要应包括本论文的目的、主要研究内容、研究方法、创造性成果及其理论与实际意义。摘要中不宜使用公式、化学结构式、图表和非公知公用的符号与术语,不标注引用文献编号,同时避免将摘要写成目录式的内容介绍。

  关键词是为了文献标引工作、用以表示全文主要内容信息的单词或术语。关键词不超过5个,每个关键词中间用分号分隔。(模板作者注:关键词分隔符不用考虑,模板会自动处理。英文关键词同理。)

\end{cabstract}

% 英文摘要
\begin{eabstract}

  Externally pressurized gas bearing has been widely used in the field of aviation, semiconductor, weave, and measurement apparatus because of its advantage of high accuracy, little friction, low heat distortion, long life-span, and no pollution. In this thesis, based on the domestic and overseas researching\dots\dots

  Key words are terms used in a dissertation for indexing, reflecting core information of the dissertation. An abstract may contain a maximum of 5 key words, with semi-colons used in between to separate one another.

\end{eabstract}


% 生成封面、中英文摘要
\makecover

% 物理量名称表,若采用标准符号则不需要此表
% \input{front/denotation}

% 中文目录
\tableofcontents

% 英文目录,本硕不要求
% \tableofengcontents

% 开始写正文
\mainmatter

% 第1章
% !TEX root = ../main.tex

% 中英标题:\chapter{中文标题}[英文标题]
\chapter{绪\hspace{1em}论}[Introduction]
% 绪论也需像摘要、目录格式那样空出两个半角字符。

\section{课题背景及研究的目的和意义}[Background, objective and significance of the subject]

% 正文内容,注意LaTeX分段有两种方法,直接空一行或者使用<\par>
% 默认首行缩进,不需要在代码编辑区手动敲空格
发展国防工业、微电子工业等尖端技术需要精密和超精密的仪器设备,精密仪器设备要求高速、\dots\dots

\dots\dots

\section{气体润滑轴承及其相关理论的发展概况}[Developmental of gas-lubricated bearing and correlated theories]

气体轴承是利用气膜支撑负荷或减少摩擦的机械构件。\dots\dots

\dots\dots

\subsection{气体润滑轴承的发展}[Developmental of gas-lubricated bearing]

1828年,R.R.Willis$^{[3]}$发表了一篇关于小孔节流平板中压力分布的文章,这是有记载的研究气体润滑的最早文献。\dots\dots

根据间隙内气膜压力的产生原理,气体轴承可以分为四种基本形式:

(1)气体静压轴承:加压气体经过节流器进入间隙,在间隙内产生压力气膜使物体浮起的气体轴承,\dots\dots

\subsubsection{气体润滑轴承的分类}[Classification of gas-lubricated bearing]

\dots\dots

\subsubsection{多孔质气体静压轴承的研究}[Research on porous externally pressurized gas bearing]

由于气体的压力低和可压缩性,\dots\dots。

\section{本文的主要研究内容}[Main research contents of this subject]

本课题的研究内容主要是针对局部多孔质止推轴承的多孔质材料的渗透
率、静压轴承的静态特性、稳定性及其影响因素进行展开,\dots\dots。


% 第2章
% !TEX root = ../main.tex

% 中英标题:\chapter{中文标题}[英文标题]
\chapter{排版图片}[Typesetting pictures]

\section{引言}[Introduction]
图应有自明性。插图应与文字紧密配合,文图相符,内容正确。选图要力求精练,插图、照
片应完整清晰。机械工程图:采用第一角投影法,严格按照GB4457~GB131-83《机械制图》
标准规定。数据流程图、程序流程图、系统流程图等按GB1526-89标准规定。电气图:图形
符号、文字符号等应符合附录3所列有关标准的规定。流程图:必须采用结构化程序并正确
运用流程框图。对无规定符号的图形应采用该行业的常用画法。坐标图的坐标线均用细实线
,粗细不得超过图中曲线;有数字标注的坐标图,必须注明坐标单位。照片图要求主题和主
要显示部分的轮廓鲜明,便于制版。如用放大或缩小的复制品,必须清晰,反差适中。照片
上应有表示目的物尺寸的标度。引用文献中的图时,除在正文文字中标注参考文献序号以外
,还必须在中、英文表题的右上角标注参考文献序号。

\section{博士毕业论文双语题注}[Doctoral picture example]

博士毕业论文双语题注如\figref{golfer1}所示。

\begin{figure}[htpb]
\centering
\includegraphics[width = 0.4\textwidth]{golfer}
\bicaption[golfer1]{}{打高尔夫球球的人(博士论文双语题注)}{Fig.$\!$}{The person playing golf (Doctoral thesis)}
\end{figure}

每个图均应有图题(由图序和图名组成),图题不宜有标点符号,图名在图序之后空1个半
角字符排写。图序按章编排,如第1章第一个插图的图号为“图1-1”。图题置于图下,硕士论
文只用中文,博士论文用中、英两种文字,居中书写,中文在上,要求中文用宋体5号字,
英文用Times New Roman 5号字。有图注或其它说明时应置于图题之上。引用图应注明出处
,在图题右上角加引用文献号。图中若有分图时,分图题置于分图之下或图题之下,可以只
用中文书写,分图号用a)、b)等表示。图中各部分说明应采用中文(引用的外文图除外)或
数字符号,各项文字说明置于图题之上(有分图时,置于分图题之上)。图中文字用宋体、
Times New Roman字体,字号尽量采用5号字(当字数较多时可用小5号字,以清晰表达为原
则,但在一个插图内字号要统一)。同一图内使用文字应统一。图表中物理量、符号用斜体
。
\subsection{本硕论文题注}[Other picture example]

本硕论文题注如\figref{fig:bm}所示。

\begin{figure}[ht]
\centering
\includegraphics[width = 0.4\textwidth]{golfer}
\caption{打高尔夫球的人,硕士论文要求只用汉语}
\label{fig:bm}
\end{figure}

\subsection{并排图和子图}[Abreast-picture and Sub-picture example]
\subsubsection{并排图}[Abreast-picture example]

使用并排图时,需要注意对齐方式。默认情况是中部对齐。这里给出中部对齐、顶部对齐
、图片底部对齐三种常见方式。其中,底部对齐方式有一个很巧妙的方式,将长度比较小
的图放在左面即可。

\lipsum[2]

\begin{figure}[htbp]
\centering
\begin{minipage}{0.4\textwidth}
\centering
\includegraphics[width=\textwidth]{golfer}
\bicaption[golfer2]{}{打高尔夫球的人}{Fig.$\!$}{The person playing golf}
\end{minipage}
\centering
\begin{minipage}{0.4\textwidth}
\centering
\includegraphics[width=\textwidth]{golfer}
\bicaption[golfer3]{}{打高尔夫球的人。注意,这里默认居中}{Fig.$\!$}{The person playing golf. Please note that, it is vertically center aligned by default.}
\end{minipage}
\end{figure}

\begin{figure}[htbp]
\centering
\begin{minipage}[t]{0.4\textwidth}
\centering
\includegraphics[width=\textwidth]{golfer}
\bicaption[golfer5]{}{打高尔夫球的人}{Fig.$\!$}{The person playing golf}
\end{minipage}
\centering
\begin{minipage}[t]{0.4\textwidth}
\centering
\includegraphics[width=\textwidth]{golfer}
\bicaption[golfer8]{}{打高尔夫球的人。注意,此图是顶部对齐}{Fig.$\!$}{The person playing golf. Please note that, it is vertically top aligned.}
\end{minipage}
\end{figure}

\begin{figure}[htbp]
\centering
\begin{minipage}[t]{0.4\textwidth}
\centering
\includegraphics[width=\textwidth,height=\textwidth]{golfer}
\bicaption[golfer9]{}{打高尔夫球的人。注意,此图对齐方式是图片底部对齐}{Fig.$\!$}{The person playing golf. Please note that, it is vertically bottom aligned for figure.}
\end{minipage}
\centering
\begin{minipage}[t]{0.4\textwidth}
\centering
\includegraphics[width=\textwidth]{golfer}
\bicaption[golfer6]{}{打高尔夫球的人}{Fig.$\!$}{The person playing golf}
\end{minipage}
\end{figure}

\subsubsection{子图}[Sub-picture example]

注意:子图题注也可以只用中文。规范规定“分图题置于分图之下或图题之下”,但没有给出具体的格式要求。
没有要求的另外一个说法就是“无论什么格式都不对”。
所以只有在一个图中有标注“(a),(b)”,无法使用\cs{subfigure}的情况下,使用最后一个图例中的格式设置方法,否则不要使用。
为了应对“无论什么格式都不对”,这个子图图题使用“minipage”和“description”环境,宽度,对齐方式可以按照个人喜好自由设置,是否使用双语子图图题也可以自由设置。

\lipsum[1-3]

无意义文字,每页底部不要留空白。

\lipsum[4-5]

\begin{figure}[!ht]
\setlength{\subfigcapskip}{-1bp}
\centering
\begin{minipage}{\textwidth}
\centering
\subfigure{\label{golfer41}}\addtocounter{subfigure}{-2}
\subfigure[The person playing golf]{\subfigure[打高尔夫球的人~1]{\includegraphics[width=0.4\textwidth]{golfer}}}
\hspace{2em}
\subfigure{\label{golfer42}}\addtocounter{subfigure}{-2}
\subfigure[The person playing golf]{\subfigure[打高尔夫球的人~2]{\includegraphics[width=0.4\textwidth]{golfer}}}
\end{minipage}
\centering
\begin{minipage}{\textwidth}
\centering
\subfigure{\label{golfer43}}\addtocounter{subfigure}{-2}
\subfigure[The person playing golf]{\subfigure[打高尔夫球的人~3]{\includegraphics[width=0.4\textwidth]{golfer}}}
\hspace{2em}
\subfigure{\label{golfer44}}\addtocounter{subfigure}{-2}
\subfigure[The person playing golf. Here, 'hang indent' and 'center last line' are not stipulated in the regulation.]{\subfigure[打高尔夫球的人~4。注意,规范中没有明确规定要悬挂缩进、最后一行居中。]{\includegraphics[width=0.4\textwidth]{golfer}}}
\end{minipage}
\vspace{0.2em}
\bicaption[golfer4]{}{打高尔夫球的人}{Fig.$\!$}{The person playing gol}
\end{figure}

\begin{figure}[t]
  \centering
  \begin{minipage}{.7\linewidth}
    \setlength{\subfigcapskip}{-1bp}
    \centering
    \begin{minipage}{\textwidth}
      \centering
      \subfigure{\label{golfer45}}\addtocounter{subfigure}{-2}
      \subfigure[The person playing golf]{\subfigure[打高尔夫球的人~1]{\includegraphics[width=0.4\textwidth]{golfer}}}
      \hspace{4em}
      \subfigure{\label{golfer46}}\addtocounter{subfigure}{-2}
      \subfigure[The person playing golf]{\subfigure[打高尔夫球的人~2]{\includegraphics[width=0.4\textwidth]{golfer}}}
    \end{minipage}
    \vskip 0.2em
  \wuhao 注意:这里是中文图注添加位置(我工要求,图注在图题之上)。
    \vspace{0.2em}
\bicaption[golfer47]{}{打高尔夫球的人。注意,此处我工有另外一处要求,子图图题可以位于主图题之下。但由于没有明确说明位于下方具体是什么格式,所以这里不给出举例。}{Fig.$\!$}{The person playing golf. Please note that, although it is appropriate to put subfigures' captions under this caption as stipulated in regulation, but its format is not clearly stated.}
  \end{minipage}
\end{figure}

\begin{figure}[t]
\centering
% \begin{tikzpicture}
% 	\node[anchor=south west,inner sep=0] (image) at (0,0) {\includegraphics[width=0.3\textwidth]{golfer}};
% 	\begin{scope}[x={(image.south east)},y={(image.north west)}]
% 		\node at (0.3,0.5) {a)};
% 		\node at (0.8,0.2) {b)};
% 	\end{scope}
% \end{tikzpicture}
\includegraphics[width=0.3\textwidth]{golfer}
\bicaption[golfer0]{}{打高尔夫球球的人(博士论文双语题注)}{Fig.$\!$}{The person playing golf (Doctoral thesis)}
\vskip -0.4em
 \hspace{2em}
\begin{minipage}[t]{0.3\textwidth}
\wuhao \setlist[description]{font=\normalfont}
	\begin{description}
		\item[(a)]子图图题
		\item[(a)]Subfigure caption
	\end{description}
 \end{minipage}
 \hspace{2em}
 \begin{minipage}[t]{0.3\textwidth}
\wuhao \setlist[description]{font=\normalfont}
	\begin{description}
		\item[(b)]子图图题
		\item[(b)]Subfigure caption
	\end{description}
\end{minipage}
\end{figure}

\begin{figure}[!ht]
	\centering
	\begin{sideways}
		\begin{minipage}{\textheight}
			\centering
			\fbox{\includegraphics[width=0.2\textwidth]{golfer}}
			\fbox{\includegraphics[width=0.2\textwidth]{golfer}}
			\fbox{\includegraphics[width=0.2\textwidth]{golfer}}
			\fbox{\includegraphics[width=0.2\textwidth]{golfer}}
			\fbox{\includegraphics[width=0.2\textwidth]{golfer}}
			\fbox{\includegraphics[width=0.2\textwidth]{golfer}}
			\fbox{\includegraphics[width=0.2\textwidth]{golfer}}
\bicaption[golfer7]{}{打高尔夫球的人(非规范要求)}{Fig.$\!$}{The person playing golf (Not stated in the regulation)}
		\end{minipage}
	\end{sideways}
\end{figure}

\clearpage

如果不想让图片浮动到下一章节,那么在此处使用\cs{clearpage}命令。

\section{如何做出符合规范的漂亮的图}

关于作图工具在后文\ref{drawtool}中给出一些作图工具的介绍,此处不多言。
此处以R语言和Tikz为例说明如何做出符合规范的图。

\subsection{Tikz作图}

使用Tikz作图核心思想是把格式、主题、样式与内容分离,定义在全局中。
注意字体设置可以有两种选择,如果字少,用五号字,字多用小五。
使用Tikz作图不会出现字体问题,字体会自动与正文一致。

\subsection{R作图}

R是一种极具有代表性的典型的作图工具,应用广泛。
与Tikz图不同,R作图分两种情况:(1)可以转换为Tikz码;(2)不可转换为Tikz码。
第一种情况图形简单,图形中不含有很多数据点,使用R语言中的Tikz包即可。
第二种情况是图形复杂,含有海量数据点,这时候不要转成Tikz矢量图,这会使得论文体积巨大。
推荐使用pdf或png非矢量图形。
使用非矢量图形时要注意选择好字号(五号或小五),和字体(宋体、新罗马)然后选择生成图形大小,注意此时在正文中使用\cs{includegraphics}命令导入时,不要像导入矢量图那样控制图形大小,使用图形的原本的
宽度和高度,这样就确保了非矢量图形中的文字与正文一致了。

为了控制\hitszthesis\ 的大小,此处不给出具体举例。

\subsection{专业绘图工具}[Processional drawing tool]
\label{drawtool}

推荐使用tikz包,使用tikz源码绘图的好处是,图片中的字体与正文中的字体一致。具体如
何使用tikz绘图不属于模板范畴。

tikz适合用来画不需要大量实验数据支撑示意图。但R语言等专业绘图工具具有画出各种、
专业、复杂的数据图。R语言中有tikz包,能自动生成tikz码,这样tikz几乎无所不能。
对于排版有极致追求的小伙伴,可以参考
\href{http://www.texample.net/tikz/resources/}{http://www.texample.net/tikz/resources/}
中所列工具,几乎所有作图软件所作的图形都可转成tikz,然后可以自由的在tikz中修改
图中内容,定义字体等等。实现前文窝工规范中要求的图中字体的一致性的终极目标。

\lipsum[1]

\section{本章小结}[Brief summary]

\lipsum[1]


% 第3章
\input{body/chapter03}

% 第4章
\input{body/chapter04}

% 第5章
\input{body/chapter05}

% 第6章
\input{body/chapter06}

% 开始写正文之后的部分
\backmatter

%%%% \begin{本科书序} %%%% 这是一个假的环境,本科请用这里的内容

% 结论
\input{back/conclusion}

% 参考文献
\bibliographystyle{hitszthesis}
\bibliography{reference}

% 发表文章
\input{back/publications}

% 授权
\authorization

% 授权页为扫描的PDF文件(scan.pdf),与上面的命令互斥
% \authorization[scan.pdf]

% 致谢
\input{back/acknowledgements}

% 附录
% 设置附录部分只包含页眉
% \SetAppendixWithOnlyHeadings
% 设置附录部分页码从1开始编号的命令在<back/appendix01.tex>里
\begin{appendix}
  \input{back/appendix01}
  \input{back/appendix02}
  \input{back/appendix03}
\end{appendix}

%%%% \end{本科书序}


%%%% \begin{硕博书序} %%%% 这是一个假的环境,硕、博请用这里的内容

% % 结论
% \input{back/conclusion}

% % 参考文献
% \bibliographystyle{hitszthesis}
% \bibliography{reference}

% % 附录
% \begin{appendix}
%   \input{back/appendixA.tex}
%   \input{back/appendixB.tex}
% \end{appendix}

% % 发表文章
% \input{back/publications}

% % 索引
% % \input{back/ceindex}

% % 授权
% \authorization

% % 授权页为扫描的PDF文件(scan.pdf),与上面的命令互斥
% % \authorization[scan.pdf]

% % 致谢
% \input{back/acknowledgements}

% % 个人简介
% \input{back/resume}

%%%% \end{硕博书序}


% 结束文档撰写
\end{document}
%%=============================================

% Local Variables:
% TeX-engine: xetex
% End:
